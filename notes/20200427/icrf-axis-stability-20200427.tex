%% file: example.tex = LaTeX + BibTex example for article-like report
%% init: sometime 1993 for my practical "Stellar Spectra A"
%% last: Jan 17 2020  Rob Rutten  Deil
%% site: http://www.staff.science.uu.nl/~rutte101/rrweb/rjr-edu/manuals/student-report/

%% First read ``latex-bibtex-simple-manual'' at
%% http://www.staff.science.uu.nl/~rutte101/Report_recipe.html

%% Then run this file to see what it does:
%%    latex example     bibtex example     latex example     latex example
%% inspect with xdvi example &, or pdflatex into example.pdf for inspection.

%% Take out my content to make a template for your report.

%%%%%%%%%%%%%%%%%%%%%%%%%%%%%%%%%%%%%%%%%%%%%%%%%%%%%%%%%%%%%%%%%%%%%%%%%%%%
\documentclass{aa-note}    %% Astronomy & Astrophysics style class aa.cls v8.2

%% load latex packages
\usepackage{graphicx,url,twoopt,natbib}
\usepackage[varg]{txfonts}           %% old-fashioned A&A font choice
\usepackage{hyperref}                %% for pdflatex
%%\usepackage[breaklinks]{hyperref}  %% for latex+dvips, not for pdflatex
%%\usepackage{breakurl}              %% for latex+dvips, not for pdflatex

%% define link colors
\hypersetup{
  colorlinks=true,   %% links colored instead of frames
  urlcolor=blue,     %% external hyperlinks
  linkcolor=blue     %% internal latex links (eg Fig)
}

\bibpunct{(}{)}{;}{a}{}{,}    %% natbib cite format used by A&A and ApJ
\pagestyle{plain}             %% undo the fancy A&A pagestyle
\let\footnotesize\tiny        %% keep this to 1 page (aa.cls sets \small)

%% Add commands to add a note or link to a reference
\makeatletter
\newcommand{\bibnote}[2]{\global\@namedef{#1note}{#2}}
\newcommand{\biblink}[2]{\global\@namedef{#1link}{#2}}
\makeatother

%% Commands to make citations ADS clickers and to add such also to refs.
%% The twoopt definition permits parameters as in natbib (which I never use).
%% The stonyslink solves a stop problem when using latex instead of pdflatex.
%% May 20 2019: switch to "new" ADS (classic in EDP/A&A readme still work)
\makeatletter
  \protected\def\stonyslink{%
     \def\hyper@linkstart##1##2{}\let\hyper@linkend\@empty}
  \newcommandtwoopt{\citeads}[3][][]{%
   \href{http://ui.adsabs.harvard.edu/abs/#3/abstract}%
        {\stonyslink \citealp[#1][#2]{#3}}%   %% Rutten, 2000
   \biblink{#3}{\href{http://ui.adsabs.harvard.edu/abs/#3/abstract}{ADS}}}
 \newcommandtwoopt{\citepads}[3][][]{%
   \href{http://ui.adsabs.harvard.edu/abs/#3/abstract}%
        {\stonyslink \citep[#1][#2]{#3}}%     %% (Rutten 2000)
   \biblink{#3}{\href{http://ui.adsabs.harvard.edu/abs/#3/abstract}{ADS}}}
 \newcommandtwoopt{\citetads}[3][][]{%
   \href{http://ui.adsabs.harvard.edu/abs/#3/abstract}%
        {\stonyslink \citet[#1][#2]{#3}}%     %% Rutten (2000)
  \biblink{#3}{\href{http://ui.adsabs.harvard.edu/abs/#3/abstract}{ADS}}}
 \newcommandtwoopt{\citeyearads}[3][][]{%
   \href{http://ui.adsabs.harvard.edu/abs/#3/abstract}%
        {\stonyslink \citeyear[#1][#2]{#3}}%  %% 2000
   \biblink{#3}{\href{http://ui.adsabs.harvard.edu/abs/#3/abstract}{ADS}}}
\makeatother

%% ADS specific page opener = {bibcode}{pdf page number}{link text}.
%% ADS promised that the "classic" link below keeps working in "new" ADS;
%% I use it because it opens ADS bibcode pdf's and ADS arXiv altcode pdf's
%% whereas "new" ADS needs separate commands (to PUB_PDF and EPRINT_PDF)
\def\linkadspage#1#2#3{\href{http://adsabs.harvard.edu/cgi-bin/nph-data_query?bibcode=#1\&link_type=ARTICLE\&db_key=AST\#page=#2}{#3}}

%% Spectral species
\def\HI{\ion{H}{I}}            %% A&A; for aastex use \def\HI{\ion{H}{1}}
\def\MgI{\ion{Mg}{I}}          %% A&A; for aastex use \def\MgI{\ion{Mg}{1}}
\def\MgII{\ion{Mg}{II}}        %% A&A; for aastex use \def\MgII{\ion{Mg}{2}}

%% Hyphenation
\hyphenation{Schrij-ver}       %% Dutch ij is a single character


%%%%%%%%%%%%%%%%%%%%%%%%%%%%%%%%%%%%%%%%%%%%%%%%%%%%%%%%%%%%%%%%%%%%%%%%%%%%
\begin{document}

%% Simple header (replacing A&A commands which produce the A&A banner)

\twocolumn[{%
\vspace*{4ex}
\begin{center}
  {\Large \bf COMPARIOSN OF OPA SOLUTIONS}\\[4ex]
  {\large \bf Niu Liu$^{1}$
  % , S\'{e}bastien Lambert. Carlsson$^2$
              % and N. G. Shchukina$^3$
              }\\[4ex]
  \begin{minipage}[t]{16cm} \small
        $^1$ School of Astronomy and Space Science,
        Key Laboratory of Modern Astronomy and Astrophysics (Ministry of Education),
        Nanjing University, Nanjing, P. R. China\\
        % $^2$ SYRTE, Observatoire de Paris, Universit\'{e} PSL, CNRS,
        % Sorbonne Universit\'{e}, LNE, Paris, France\\
        % $^3$ Main Astronomical Observatory,
        %      252127 Kiev, Ukraine\\

  {\bf Abstract.~}
   \vspace*{2ex}
  \end{minipage}
\end{center}
}]


%%%%%%%%%%%%%%%%%%%%%%%%%%%%%%%%%%%%%%%%%%%%%%%%%%%%%%%%%%%%%%%%%%%%%%%%%%%%
\section{Introduction}     \label{sec:introduction}
%%%%%%%%%%%%%%%%%%%%%%%%%%%%%%%%%%%%%%%%%%%%%%%%%%%%%%%%%%%%%%%%%%%%%%%%%%%%
% The existence of two emission features in the solar spectrum near
% 12~$\mu$m was announced by
% %% Note tilde, mandatory to avoid line break between number and unit
% \citetads{1981ApJ...247L..97M}%% Murcray+others MgI features,
% %% Add a brief comment as identifier for yourself and co-authors.
% %% Using ADS bibcodes is mandatory for my clickable cites but is also useful
% %% to avoid confusing co-authors with your own inventions
% \footnote{Depending on your pdf-viewer settings, clicking a name-year
%   citation may open the corresponding ADS abstract page in your browser.},
% but only when they were informed by L.~Testerman and J.~Brault that
% %% Note tildes to avoid line breaks
% they had noticed them too.
% Before that, \citetads{1980STIN...8031298G} %% Goldman+others IR atlas
% had white-pasted them out of their spectrum atlas in the mistaken
% belief that all solar and telluric lines should be in absorption.
% We explained these emission features many years ago
% (\citeads{1992A&A...253..567C}, % Carlsson+Rutten+Shchukina MgI
% henceforth Pub~I; see also \citeads{1994IAUS..154..309R}).

%% Add (Pub I) to the reference
% \bibnote{1992A&A...253..567C}{(Pub~I)}
%% Don't use ``Paper~I'' since journals appear electronically only

%% Define \PubI as an ADS clicker
% \def\PubI{\href{http://adsabs.harvard.edu/abs/1992A&A...253..567C}{Pub~I}}

%% Note: citations are properly broken by the breaklinks hyperref option
%% when using dvips, but then do not link in the Adobe pdf reader.
%% Citation breaks are handled correctly by pdflatex.

%%%%%%%%%%%%%%%%%%%%%%%%%%%%%%%%%%%%%%%%%%%%%%%%%%%%%%%%%%%%%%%%%%%%%%%%%%%%
\section{Model computations}    \label{sec:computations}
%%%%%%%%%%%%%%%%%%%%%%%%%%%%%%%%%%%%%%%%%%%%%%%%%%%%%%%%%%%%%%%%%%%%%%%%%%%

\subsection{Background}
%%%%%%%%%%%%%%%%%%%%%%%%%
% In the solar photosphere NLTE departure diffusion occurs in the upper
% reaches of the \MgI\ term structure (see
% \linkadspage{1992A&A...253..567C}{7}{Figs.~3 and 4}\footnote{%
%   Links to cited figures or equations may open the pertinent page
%   via ADS (except in macOS Preview which shunts to the start page).}
% of \PubI).
% It is akin to optically-thin collisional-radiative recombination along
% Rydberg levels in tenuous plasmas as sketched in
% %% The \, below is a thinner space than a tilde; can also be good for units
% Fig.\,\ref{fig:waterfalls}\footnote{%
%   Multi-panel figure production: prepare separate figures, each with
%   full axis annotation, and paste them together using the latex commands
%   in my
%   \href{http://www.staff.science.uu.nl/~rutte101/rrweb/rjr-edu/manuals/student-report/cutmultipanel.tex}{{\tt cutmultipanel.tex}}.
%   These remove superfluous axis annotation between adjacent panels and
%   rescale them to the same size; they also maintain image resolution for
%   zoom-in per pdf viewer.
%   This way you can define the multi-panel assembly layout (e.g.,
%   column-wide vertical stacking or page-wide horizontal) while writing,
%   including re-ordering figures from coauthors.
% }.
% The righthand cartoon in \linkadspage{2019arXiv190804624R}{4}{Fig.~2
% here}\footnote{Link to a figure page in the arXiv pdf for a Springer
%   publication not directly accessible at ADS.}
% shows the \HI\ Rydberg ladder more formally.



%% {fig:waterfalls}
%===========================================================================
% \begin{figure}[hbtp]
%   \centering
%   \includegraphics[width=\columnwidth]{waterfalls}  %% no file extension
%   \caption[]{\label{fig:waterfalls} %
%   Collisional-radiative recombination along \MgI\ Rydberg states
%   visualized by Mats Carlsson for drop-pool kayakers.
%   {\em Left:\/} the largest recombination flow from the magnesium
% %% Always end italics strings with \/
%   population reservoir in the \MgII\ ground state is into the highest
%   \MgI\ level ($n\!=\!9$ in the present model).
% %% The ! signs above reduce the spacing around the =
%   {\em Right:\/} along the $\Delta n\!=\!1$ downward ladder the flow
%   is initially dominated by collisional transitions but radiative
%   transitions take over lower down.
%   The recombination flow is driven by photon losses in strong \MgI\
%   lines and is balanced by radiative ionization in ultraviolet \MgI\
%   edges.
%   Similar Rydberg flows occur in other elements, but the \MgI\ Rydberg
%   levels contain the largest photospheric populations, exceeding even
%   the \HI\ ones.
%   From \citetads{1994IAUS..154..309R}. % Rutten+Carlsson Tucson review
%   }
% \end{figure}
% ===========================================================================
%% Have "floats" such as figures between blank lines to make them float


%% In this file every sentence starts on a new line for better
%% readability, easier change, better synchronization in
%% shared-access systems such as svn, easier use of (mg)diff.
%% This is automatically formatted by emacs in the setup I use,
%% specified in my ~/.emacs shown in "Recipes for linux/unix"
%% on my website.


\subsection{Method}
%%%%%%%%%%%%%%%%%%%%
% We solved the statistical equilibrium and radiative transfer equations
% for all relevant levels and frequencies in \MgI\ and \MgII\ for
% various models of the solar atmosphere, including the standard one
% formulated in the monumental articles by Vernazza et al.\
% (\citeyearads{1973ApJ...184..605V}, % VALI
% \citeyearads{1976ApJS...30....1V}, % VALII
% \citeyearads{1981ApJS...45..635V}). % VALIII
%% Example same-authors multiple citation list

    \begin{table}
        \centering
        \caption{
        Orientation offsets with referred to the ICRF3 S/X-band frame.
        }
        \begin{tabular}{ccccc}
            \hline \noalign{\smallskip}
            & $N_{\rm sou,def}$ & $R_X$ & $R_Y$ & $R_Z$  \\
            &  &$\mathrm{\mu as}$ & $\mathrm{\mu as}$ & $\mathrm{\mu as}$  \\
            \noalign{\smallskip}
            \hline
            \noalign{\smallskip}
            ICRF2       & 296  & $ -10$  $\pm$ 7 & $ +20$  $\pm$  8 & $  -1$  $\pm$  7 \\
            ICRF3-K     & 193  & $ -10 \pm 10$   & $ -9 \pm  11$ & $ -5$  $\pm$  7 \\
            ICRF3-X/Ka  & 176  & $ -28$  $\pm$ 26 & $ -25$  $\pm$ 27 & $  -52$  $\pm$ 19 \\
            \noalign{\smallskip}
            \hline
        % OPA2019a & 4380 & 2920 & $ +28$  $\pm$  2 & $ -50$  $\pm$  2 & $  -1$  $\pm$  1 \\
        \end{tabular}
    \end{table}

    \begin{table}
        \centering
        \caption{
        Orientation offsets with referred to the ICRF3 S/X-band frame from bootstrap sampling.
        }
        \begin{tabular}{cccccc}
            \hline \noalign{\smallskip}
            & $N_{\rm com}$ & $N_{\rm used}$ & $R_X$ & $R_Y$ & $R_Z$  \\
            &  &  & $\mathrm{\mu as}$ & $\mathrm{\mu as}$ & $\mathrm{\mu as}$  \\
            \noalign{\smallskip}
            \hline
            \noalign{\smallskip}
            ICRF2       & 3410 & 2275 & $ -19$  $\pm$  6 & $ +22$  $\pm$  5 & $  +0$  $\pm$  4 \\
            ICRF3-K     & 793  & 530  & $ -21$  $\pm$  8 & $ -18$  $\pm$  8 & $ -11$  $\pm$  6 \\
            ICRF3-X/Ka  & 638  & 425  & $ -53$  $\pm$ 19 & $  -7$  $\pm$ 19 & $  +6$  $\pm$ 10 \\
            % OPA2019a    & 4380 & 2920 & $ +28$  $\pm$  2 & $ -50$  $\pm$  2 & $  -1$  $\pm$  1 \\
            \noalign{\smallskip}
            \hline
        \end{tabular}
    \end{table}

%%%%%%%%%%%%%%%%%%%%%%%%%%%%%%%%%%%%%%%%%%%%%%%%%%%%%%%%%%%%%%%%%%%%%%%%%%%%
\section{Conclusion} \label{sec:conclusion}
%%%%%%%%%%%%%%%%%%%%%%%%%%%%%%%%%%%%%%%%%%%%%%%%%%%%%%%%%%%%%%%%%%%%%%%%%%%%
% Our computation explained the formation of the enigmatic
% \MgI\,12\,$\mu$m emission features.
% They arise through population depletion by line photon losses and
% population replenishment from the ionic reservoir through highly
% excited levels.
% A Rydberg-channel replenishment flow is realized by
% collisionally-dominated population diffusion via ladder-wise departure
% divergence (see Fig.~\ref{fig:waterfalls} in
% Sect.~\ref{sec:computations}).
%% Note the tilde.  Fig. \ref would generate too much end-of-sentence space
%% and a line break between Fig. and the number is undesirable

%%%%%%%%%%%%%%%%%%%%%%%%%%%%%%%%%%%%%%%%%%%%%%%%%%%%%%%%%%%%%%%%%%%%%%%%%%%%
\begin{acknowledgements}
  % We are indebted to NASA's \href{http://ui.adsabs.harvard.edu}{ADS}
  % for its magnificent literature and bibliography serving which
  % was severely missed by us -- fortunately unknowingly --
  % in pre-internet 1992.
%% us -- fortunately is A&A style; ApJ wants us---unfortunately.
\end{acknowledgements}

%%%%%%%%%%%%%%%%%%%%%%%%%%%%%%%%%%%%%%%%%%%%%%%%%%%%%%%%%%%%%%%%%%%%%%%%%%%%
%% references
\bibliographystyle{aa-note} %% aa.bst but adding links and notes to references
%%\raggedright              %% for latex+dvips, not for pdflatex
\bibliography{example}      %% example.bib = bibtex entries copied from ADS

\end{document}
